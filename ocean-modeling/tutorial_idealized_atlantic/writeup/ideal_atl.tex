\documentclass[a4paper,11pt]{article}
\usepackage[utf8]{inputenc}
\usepackage{geometry,amsmath,color,graphicx,subfig}
\geometry{margin=1in}


%opening
\title{\vspace{-10ex}100 Year Idealized Atlantic}
\author{Tim Smith}
\date{\vspace{-3ex}}

%% --- Figures
\DeclareGraphicsExtensions{.png,.pdf}
\newcommand{\degSym}{$^{\circ}$}

\begin{document} 

%\section{Original Wind Stress}
%	First is the output from running 100 years with the original set up. 

%%%%%%%%%%%%%%%%
% Original set up
%%%%%%%%%%%%%%%%

\begin{figure}
\centering
\includegraphics[width=\textwidth]{../run-orig/GlobalAvg_pTau}
\caption{\textbf{Original Setup}. Global mean Sea Surface Temperature (top) and Heat Content (bottom) as yearly time averaged quantities. The global averages are computed as volume weighted averages. Here the original wind field is used. The SST seems to stablize long before Heat Content. It seems like global heat content needs another 50-100 years to truly reach equilibrium. }
\label{fig:globalAvgs}
\end{figure}

\begin{figure}
\centering
\includegraphics[width=\textwidth]{../run-orig/vol_snapshots}
\caption{\textbf{Original Setup}. Snapshots of temperature (left), salinity (middle left), zonal velocity (middle right), and vertical velocity (right) at 20m depth (top row) and 1350m depth (bottom row). Snapshot taken at the end of 100, 360 day years. Here the original wind field is used. The surface level temperature and salinity plots show a symmetric ``Gulf stream'', where warm salty waters are confined to $\sim 30-40^{\circ}N/S$. There are two eastward equatorial currents present in the horizontal velocity field, and westward currents from $\sim20-60^{\circ}N/S$. These flows resemble the subtropical gyres in the Northern and Southern Hemispheres. The vertical flow shows nearly no motion at the surface. At 1350m the temperature field is relatively uniform, with warm spots around $\sim50^{\circ}N/S$, which coincides with the downwelling sites in the vertical velocity field. Thus, it seems that these warm patches are from warmer waters at the surface circulating to depth to travel back to the equator. Oddly the surface vertical velocity has two very small points of downwelling. Although downwelling is expected at around $60^{\circ}N$, these points seem nonphysical.}
\label{fig:volSnaps}
\end{figure}

%%%%%%%%%%%%%%%%
% 2f set up
%%%%%%%%%%%%%%%%

\begin{figure}
\centering
\includegraphics[width=\textwidth]{../run-2f/GlobalAvg_pTau}
\caption{\textbf{Double rotation rate}. Same as in Fig. \ref{fig:globalAvgs}}
\label{fig:2f_globalAvgs}
\end{figure}

\begin{figure}
\centering
\includegraphics[width=\textwidth]{../run-2f/vol_snapshots}
\caption{\textbf{Double rotation rate}. Same as in Fig. \ref{fig:volSnaps}. Note }
\label{fig:2f_volSnaps}
\end{figure}

%%%%%%%%%%%%%%%%
% 4f set up
%%%%%%%%%%%%%%%%

\begin{figure}
\centering
\includegraphics[width=\textwidth]{../run-4f/GlobalAvg_pTau}
\caption{\textbf{Quadruple rotation rate}. Same as in Fig. \ref{fig:globalAvgs}}
\label{fig:4f_globalAvgs}
\end{figure}

\begin{figure}
\centering
\includegraphics[width=\textwidth]{../run-4f/vol_snapshots}
\caption{\textbf{Quadruple rotation rate}. Same as in Fig. \ref{fig:volSnaps}. Note }
\label{fig:4f_volSnaps}
\end{figure}

%%%%%%%%%%%%%%%%
% 8f set up
%%%%%%%%%%%%%%%%

\begin{figure}
\centering
\includegraphics[width=\textwidth]{../run-8f/GlobalAvg_pTau}
\caption{\textbf{8 x rotation rate}. Same as in Fig. \ref{fig:globalAvgs}}
\label{fig:8f_globalAvgs}
\end{figure}

\begin{figure}
\centering
\includegraphics[width=\textwidth]{../run-8f/vol_snapshots}
\caption{\textbf{8 x rotation rate}. Same as in Fig. \ref{fig:volSnaps}. Note }
\label{fig:8f_volSnaps}
\end{figure}


%%%%%%%%%%%%%%%%
% Flipped wind set up
%%%%%%%%%%%%%%%%

\begin{figure}
\centering
\includegraphics[width=\textwidth]{../run-flip/GlobalAvg_pTau}
\caption{\textbf{Flipped horizontal wind, original rotation rate}. Same as in Fig. \ref{fig:globalAvgs}}
\label{fig:flipGlobalAvgs}
\end{figure}

\begin{figure}
\centering
\includegraphics[width=\textwidth]{../run-flip/vol_snapshots}
\caption{\textbf{Flipped horizontal wind, original rotation rate}. Same as in Fig. \ref{fig:volSnaps}. Note }
\label{fig:flipVolSnaps}
\end{figure}

%%%%%%%%%%%%%%%%
% Flipped 4 x wind set up
%%%%%%%%%%%%%%%%

\begin{figure}
\centering
\includegraphics[width=\textwidth]{../run-flip4/GlobalAvg_pTau}
\caption{\textbf{Flipped and quadrupled horizontal wind, original rotation rate}. Same as in Fig. \ref{fig:globalAvgs}}
\label{fig:flipr_GlobalAvgs}
\end{figure}

\begin{figure}
\centering
\includegraphics[width=\textwidth]{../run-flip4/vol_snapshots}
\caption{\textbf{Flipped and quadrupled horizontal wind, original rotation rate}. Same as in Fig. \ref{fig:volSnaps}. Note }
\label{fig:flip4_olSnaps}
\end{figure}

%%%%%%%%%%%%%%%%
% Flipped wind and rotation set up
%%%%%%%%%%%%%%%%

\begin{figure}
\centering
\includegraphics[width=\textwidth]{../run-flip-all/GlobalAvg_pTau}
\caption{\textbf{Flipped horizontal wind and rotation rate}. Same as in Fig. \ref{fig:globalAvgs}}
\label{fig:flip_all_GlobalAvgs}
\end{figure}

\begin{figure}
\centering
\includegraphics[width=\textwidth]{../run-flip-all/vol_snapshots}
\caption{\textbf{Flipped horizontal wind and rotation rate}. Same as in Fig. \ref{fig:volSnaps}. Note }
\label{fig:flip_all_volSnaps}
\end{figure}

\end{document}

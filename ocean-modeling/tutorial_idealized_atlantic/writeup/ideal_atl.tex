\documentclass[a4paper,11pt]{article}
\usepackage[utf8]{inputenc}
\usepackage{geometry,amsmath,color,graphicx,subfig}
\geometry{margin=1in}


%opening
\title{\vspace{-10ex}100 Year Idealized Atlantic}
\author{Tim Smith}
\date{\vspace{-3ex}}

%% --- Figures
\DeclareGraphicsExtensions{.png,.pdf}
\newcommand{\degSym}{$^{\circ}$}

\begin{document} 

%\section{Original Wind Stress}
%	First is the output from running 100 years with the original set up. 

\begin{figure}
\centering
\includegraphics[width=.8\textwidth]{../orig-tau/GlobalAvg_pTau}
\caption{Global mean Sea Surface Temperature (top) and Heat Content (bottom) as yearly time averaged quantities. The global averages are computed as volume weighted averages. Here the original wind field is used. The SST seems to stablize long before Heat Content. It seems like global heat content needs another 50-100 years to truly reach equilibrium. }
\label{fig:globalAvgs}
\end{figure}

\begin{figure}
\centering
\includegraphics[width=.8\textwidth]{../orig-tau/vol_snapshots}
\caption{Snapshots of temperature (left), salinity (middle left), zonal velocity (middle right), and vertical velocity (right) at 20m depth (top row) and 1350m depth (bottom row). Snapshot taken at the end of 100, 360 day years. Here the original wind field is used. The surface level temperature and salinity plots show a symmetric ``Gulf stream'', where warm salty waters are confined to $\sim 30-40^{\circ}N/S$. There are two eastward equatorial currents present in the horizontal velocity field, and westward currents from $\sim20-60^{\circ}N/S$. These flows resemble the subtropical gyres in the Northern and Southern Hemispheres. The vertical flow shows nearly no motion at the surface. At 1350m the temperature field is relatively uniform, with warm spots around $\sim50^{\circ}N/S$, which coincides with the downwelling sites in the vertical velocity field. Thus, it seems that these warm patches are from warmer waters at the surface circulating to depth to travel back to the equator. Oddly the surface vertical velocity has two very small points of downwelling. Although downwelling is expected at around $60^{\circ}N$, these points seem nonphysical.}
\label{fig:volSnaps}
\end{figure}

\begin{figure}
\centering
\subfloat[SSH Anomaly]{{\includegraphics[width=.45\textwidth]{../orig-tau/EtaSnap}}}
\qquad
\subfloat[Horizontal Wind Field]{{\includegraphics[width=.45\textwidth]{../orig-wind}}}
\caption{(Left) Snapshot of $\eta$, the sea surface height anomaly, taken at the end of 100, 360 day years. (Right) Horizontal wind field under the original configuration. Similar to the temperature field, the SSH anomaly field is elevated at mid latitudes and decreases sharply from $\sim30^{\circ}N/S$ to $50^{\circ}N/S$. This behavior is similar to the actual gulf stream.} 
\label{fig:etaSnap}
\end{figure}

%\section{Flipped Wind Stress}
%
%	These next plots show the results after flipping the sign of the horizontal wind stress. 

\begin{figure}
\centering
\includegraphics[width=.8\textwidth]{../flip-tau/GlobalAvg_pTau}
\caption{Global mean Sea Surface Temperature (top) and Heat Content (bottom) as yearly time averaged quantities. The global averages are computed as volume weighted averages. The horizontal wind field is reversed here.}
\label{fig:flipGlobalAvgs}
\end{figure}

\begin{figure}
\centering
\includegraphics[width=.8\textwidth]{../flip-tau/vol_snapshots}
\caption{Snapshots of temperature (left), salinity (middle left), zonal velocity (middle right), and vertical velocity (right) at 20m depth (top row) and 1350m depth (bottom row). Snapshot taken at the end of 100, 360 day years. Here the horizontal wind field is reversed. The quantities at 1350m depth are very similar to the original experiment. At the surface, the horizontal velocity maintains mostly a similar structure, indicating that the wind field is not solely responsible for this field and presumably rotation dominates. However, for longitudes $>40^{\circ}W$ the horizontal velocity is near zero because the wind direction cancels with rotational effects. As a result, the temperature and salinity fields are also ``clumped'' with higher values in this region. Again there is a seemingly nonphysical point of upwelling at about $40^{\circ}N$, although the magnitude of vertical velocity here is tiny.}
\label{fig:flipVolSnaps}
\end{figure}

\begin{figure}
\centering
\subfloat[SSH Anomaly]{{\includegraphics[width=.45\textwidth]{../flip-tau/EtaSnap}}}
\qquad
\subfloat[Horizontal Wind Field]{{\includegraphics[width=.45\textwidth]{../neg-wind}}}
\caption{(Left) Snapshot of $\eta$, the sea surface height anomaly, taken at the end of 100, 360 day years. (Right) Reversed horiziontal wind field. The SSH anomaly field is ``clumped'' in the equatorial region where horizontal velocity is near zero, just as in the temperature and salinity surface fields.} 
\label{fig:flipEtaSnap}
\end{figure}

\end{document}

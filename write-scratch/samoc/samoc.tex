\documentclass[a4paper,11pt]{article}
\usepackage[utf8]{inputenc}
\usepackage{geometry,amsmath,color,graphicx}
%opening
\title{\vspace{-10ex}Computing AMOC}
\author{Tim Smith}
\date{\vspace{-3ex}}

\newcommand{\red}[1]{\textcolor{red}{#1}}

%% --- Figures
\graphicspath{ {plots/} }
\DeclareGraphicsExtensions{.png,.pdf}

\begin{document}

\maketitle

\begin{abstract}
 I have been running short model simulations, computing AMOC at 25$^{\circ}$N so I could verify my values with those in e.g. \cite{wunschAndHeimbach_AMOC} or \cite{mcCarthy}, but I'm running into some problems. Essentially, I'm not getting values that make sense. Here are some of my questions, I tried to keep it very brief. 
\end{abstract}


\section{AMOC Objective Function}

  Following the definition set out in \cite{wunschAndHeimbach_AMOC}, the monthly mean meridional volume transport northward at a given time $t$, longitude $\lambda$, latitude $\phi$, and depth $z$ is $\bar{V}(\lambda,z,\phi,t)$. With a discrete numerical integration at each grid cell, I am computing this volume transport in a given grid cell grid cell $(\phi_i,\lambda_j,z_k)$ as: 
  \begin{align}
    \bar{V}(\phi_i,\lambda_j,z_k,t) &= \int_{z_k}\int_{\lambda_j}v \,d\lambda dz \\
				    &\simeq v_i \,  \triangle \lambda_j \triangle z_k
    \label{eq:volTrans}
  \end{align}
  
  where $v_i$ is the meridional velocity defined on the Southern face of the grid cell. The monthly mean volume transport across a latitude band $\phi_i$ at depth $z_k$ from coast to coast is approximated by the sum: 
  
  \begin{equation}
    \bar{V}_{cc}(\phi_i,z_k,t) = \sum_{j}\bar{V}(\phi_i,\lambda_j,z_k,t) .
    \label{eq:vcc}
  \end{equation}

  With these definitions, AMOC is the maximum transport: 
  
  \begin{equation}
    \bar{V}_{MOC}(z_{max}(t),\phi,t) = \max_z\int_{z(\phi,t)}^{\eta(\phi,t)}\bar{V}_{cc}(\phi,z,t) \, dz
    \label{eq:amoc}
  \end{equation}

  \red{Question: Wouldn't this leave the units of $ \bar{V}_{MOC}(z_{max}(t),\phi,t)$ to be [Sv m]? Is $\bar{V}_{cc}(\phi_i,z_k,t)$ actually in units Sv (m depth)$^{-1}$?}
  
 % When I plot the zonally integrated quantity $\bar{V}_{cc}(\phi_i,z_k,t)$ as defined in eqn. \ref{eq:vcc} I get the plot shown in Fig. \ref{fig:Vcc}. Each line represents a monthly mean for $\bar{V}_{cc}(\phi_i,z_k,t)$ from a 4 year run. \red{Some of the transports are huge at depths around ..., whereas the plot in \cite{wunschAndHeimbach_AMOC} (copied here in Fig. \ref{whplot}) shows these should be much smaller. Do you have any intuition for why this is so far off?}
  
  %\begin{figure}
   %\centering
 %  \includegraphics[width=\textwidth]{vcc}
 %  \caption{Each line represents a zonally integrated, monthly mean volume transport $\bar{V}_{cc}(\phi_i,z_k,t)$ from a 4 year run. The amplitudes seem to be much larger than they should be for depths below 3000m. Compare to \ref{fig:whplot} at 25$^{\circ}$N.}
 %  \label{fig:Vcc}
 % \end{figure}
  
  %\begin{figure}
  % \centering
  % \includegraphics{whplot}
  % \label{fig:whplot}
  %\end{figure}


  
  
\section{Latitude masks}
  
  When creating my mask for the objective function, I grab points where latitude is the closest latitude to 25$^{\circ}$N. However, when I plot this I get a mismatch between the different 'faces' on the grid (see next pg). I suspect this is just an artefact of plotting because the right face has more latitude grid cells (F5 range: [-90,71.62] Latitude) than the left face (F1 range: [-88.02, 72.04] Latitude). But I just wanted to check.
  
  \begin{figure}
   \centering
   \includegraphics[width=\textwidth]{latitudeMismatch}
   \label{fig:latitudeMismatch}
   \caption{There seems to be a mismatch between faces 1 \& 5 (the abrupt change is where these two faces come together).}
  \end{figure}


  
\begin{thebibliography}{3}

  \bibitem{wunschAndHeimbach_AMOC}
  Wunsch, Carl, and Patrick Heimbach. ``Two decades of the Atlantic meridional overturning circulation: Anatomy, variations, extremes, prediction, and overcoming its limitations." Journal of Climate 26.18 (2013): 7167-7186.
  
  \bibitem{mcCarthy}
  McCarthy, G. D., et al. ``Measuring the Atlantic meridional overturning circulation at 26 N." Progress in Oceanography 130 (2015): 91-111.

\end{thebibliography}

\end{document}

\documentclass[a4paper,11pt]{article}
\usepackage[utf8]{inputenc}
\usepackage{geometry,amsmath,color,graphicx}
\geometry{margin=1in}


%opening
\title{\vspace{-10ex}Forward Experiment Setup: The effect of heat flux perturbations}
\author{Tim Smith}
\date{\vspace{-3ex}}

%% --- New commands
\newcommand{\pderiv}[3][]{% \pderiv[<order>]{<func>}{<var>} 
  \ensuremath{\frac{\partial^{#1} {#2}}{\partial {#3}^{#1}}}}
  
\newcommand{\red}[1]{\textcolor{red}{#1}}
\newcommand{\degSym}{$^{\circ}$}

%% --- Figures
\graphicspath{ {../../samoc/figures/} }
\DeclareGraphicsExtensions{.png,.pdf}

\begin{document}

After computing AMOC sensitivities at 26N and 34S there are a number of questions which remain. First and foremost, the time response of the sensitivities looks very different from Pillar and Czeschel. These studies show an oscillatory behavior going backward in time, such that after roughly 15 years the sensitivity of AMOC to heat flux flips from negative to positive. This means that an increase in ocean heating will cause a decrease in the AMOC over a fifteen year time frame. After this point, the perturbation will cause an increase in the AMOC. 

Using the ECCO version 4 release 2 set up, I do not see this oscillatory behavior in heat flux sensitivity. Instead, the sensitivity looks more like a damped harmonic oscillator which converges to a negative value \red{Need to check the signs here ... }. There are a few differences in model set up which may cause these different behaviors. First, in these two papers, the model is forced with a monthly climatology of wind stress and heat and fresh water fluxes. In the ECCO set up we use smoothed forcing data based on the observed atmospheric state from 1992-2011. Secondly, the configurations in Pillar and Czeschel use SST and SSS climatological relaxation terms. Czeschel additionally uses similar relaxation terms at the Nordic sea, Weddell sea, and Strait of Gibraltar. The ECCO set up does not use any relaxation terms. These terms could cause 

\end{document}

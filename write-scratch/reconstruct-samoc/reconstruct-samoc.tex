\documentclass[a4paper,11pt]{article}
\usepackage[utf8]{inputenc}
\usepackage{geometry,amsmath,color,graphicx}
\geometry{margin=1in}


%opening
\title{\vspace{-10ex}Offline SAMOC Reconstruction}
\author{Tim Smith}
\date{\vspace{-3ex}}

%% --- New commands
\newcommand{\pderiv}[3][]{% \pderiv[<order>]{<func>}{<var>} 
  \ensuremath{\frac{\partial^{#1} {#2}}{\partial {#3}^{#1}}}}
  
\newcommand{\red}[1]{\textcolor{red}{#1}}
\newcommand{\degSym}{$^{\circ}$}

%% --- Figures
\graphicspath{ {../../samoc/figures/} }
\DeclareGraphicsExtensions{.png,.pdf}

\begin{document}

\maketitle
  \section{Variability reconstruction}
  \label{linearSensitivity}
  
  We define an objective function $\mathcal{J}$ as the monthly mean value of AMOC as in Eqn. \ref{eq:amoc}. Using the adjoint mode of the MITgcm, we compute the sensitivity of the objective function to external forcing: $\pderiv{\mathcal{J}}{\mathcal{F}}$. In this case, $\mathcal{F}$ consists of the near surface air temperature, humidity, wind stress, precipitation, and radiative forcing.  
	
  We take the objective function to be split into its 20 year mean and monthly anomaly components: 
	\begin{equation}
	  \mathcal{J}_i(t) = \bar{\mathcal{J}_i}(t) + \mathcal{J}_i'(t) \qquad i \in [1, 2, ... 12]
	\end{equation}

  Here the subscript $i$ denotes the month in which the cost function is evaluated (January, February, etc). We hypothesize that the monthly anomaly in AMOC, $\mathcal{J}_i'(t)$, is linearly dependent on external forcing such that the AMOC anomaly for a given month, $i$, can be reconstructed offline via the convolution integral:
 
	\begin{equation}
	  \mathcal{J}_i'(t) = \int_{1992}^{t}\int_x \int_y\pderiv{\mathcal{J}_i}{\mathcal{F}}(x,y,\tau-t)\mathcal{F}'(x,y,\tau)\, dxdyd\tau  
	\end{equation}

  
  \subsection{Sensitivity details}
  
  Discussing this in terms of an example. Consider a 6 month run, monthly mean values of $\pderiv{\mathcal{J}}{\mathcal{F}}$ are computed via the control package. This results in a file with 7 records, corresponding to the points in the forward model where the control fields are read in (see fig. \ref{fig:timeline}). At a given point in time during the forward mode, the field \textit{genfld} gets modified by its two neighboring controls, \textit{xxgen0,xxgen1} via linear interpolation:

	\begin{equation}
	  genfld(x,y,z,t) = \alpha xxgen0 + (1-\alpha) xxgen1
	\end{equation}

  where $\alpha=1$ at the start of the month, 0 at the end. This is performed in the subroutine \textit{ctrl\_get\_gen}.

  In the reverse mode, the adjoint sensitivity at a given point in time is given by \textit{genfld\_ad}. The monthly mean sensitivities are written to disk as active variables \textit{xxgen0\_ad,xxgen1\_ad}, computed the reverse of the above equation: 

	\begin{equation}
	  xxgen0\_ad = \alpha genfld\_ad(x,y,z,t)
	\end{equation}
	\begin{equation}
	  xxgen1\_ad = (1-\alpha) genfld\_ad(x,y,z,t) 
	\end{equation}

   
  This is performed in \textit{ctrl\_get\_gen\_ad}, where \textit{adactive\_file\_read} in reverse mode writes the active variable to disk. \red{My question is: running the code in reverse mode shows that ``adxx'' files are written to when $\alpha = 1$, but the variable that is written to disk is \textit{xxgen1\_ad}. Why is this the case? Wouldn't this result in a file full of zeros?}

  \red{Also, 
  
  

\end{document} 
